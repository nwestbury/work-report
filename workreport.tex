%%%%%%%%%%%%%%%%%%%%%%%%%%%%%%%%%%%%%%%%%%%%%%%%%%%%%%%%%%%%%%%%%%%%%
% LaTeX Template: Project Titlepage Modified (v 0.1) by rcx
%
% Original Source: http://www.howtotex.com
% Date: February 2014
% 
% This is a title page template which be used for articles & reports.
% 
% This is the modified version of the original Latex template from
% aforementioned website.
% 
%%%%%%%%%%%%%%%%%%%%%%%%%%%%%%%%%%%%%%%%%%%%%%%%%%%%%%%%%%%%%%%%%%%%%%

\documentclass[12pt]{report}
\usepackage[a4paper]{geometry}
\usepackage{fancyhdr}
\usepackage{lastpage}
\usepackage{graphicx, wrapfig, subcaption, setspace, booktabs}
\usepackage[T1]{fontenc}
\usepackage[font=small, labelfont=bf]{caption}
\usepackage{fourier}
\usepackage[protrusion=true, expansion=true]{microtype}
\usepackage[english]{babel}
\usepackage{sectsty}
\usepackage{url, lipsum}
\usepackage{enumitem}


\newcommand{\HRule}[1]{\rule{\linewidth}{#1}}
\renewcommand{\headrulewidth}{0pt}
\linespread{1.5}

%-------------------------------------------------------------------------------
% HEADER & FOOTER
%-------------------------------------------------------------------------------
% \pagestyle{fancy}
% \fancyhf{}
% \fancyhead[L]{Student ID: 20550430}
% \fancyhead[R]{University of Waterloo}
% \fancyfoot[C]{\thepage} % / \pageref{LastPage}

%-------------------------------------------------------------------------------
% TITLE PAGE
%-------------------------------------------------------------------------------

\begin{document}
\begin{titlepage}
   \begin{center}
    	\normalsize \textbf{\uppercase{University of Waterloo}} \\
		Faculty of Mathematics \\
	\end{center}	
		\vspace*{\stretch{0.1}}
	\begin{center}
		\HRule{0.5pt}
   		\LARGE \textbf{\uppercase{Scaling Systems}}
   		\HRule{0.5pt}
	\end{center}
	\vspace*{\stretch{0.1}}
	\begin{center}
	   		\normalsize {Arb Labs\\ Niagara Falls, Ontario}	
	 
	\end{center}
	\vspace*{\stretch{0.1}}
	\begin{center}
	   		\normalsize {Prepared by\\
				Nicholas Westbury\\
				3A Computer Science\\
				ID 20550430\\
	   		 	\today
	   		 }
	\end{center}
\end{titlepage}

\newpage\noindent\thispagestyle{empty}
\LARGE\textbf{\uppercase{MEMORANDUM}}\\ \\
\normalsize
To: SUPERVISOR\\
From: Nicholas\\
Date: \today\\
Re: Work Report: Scaling Systems\\
\HRule{1.5pt}\\

As we agreed, I have prepared the enclosed report, “LAN in the Workplace,” for
my 3A work report and for the Internal Computer Support Department. This
report, the second of four work reports that the Co-operative Education Program
requires that I successfully complete as part of my BMath Co-op degree
requirements, has not received academic credit.
The Internal Computer Support team that you lead provides LAN and hardware
support to internal users. My job as LAN Support Assistant required that I
handle all requests for support regarding the network. I responded to telephone,
e-mail, and in-person queries. This report is an in-depth study of the
implementation of the company’s new Local Area Network.
The Faculty of Mathematics requests that you evaluate this report for command
of topic and technical content/analysis. Following your assessment, the report,
together with your evaluation, will be submitted to the Math Undergrad Office
for evaluation on campus by qualified work report markers. The combined
marks determine whether the report will receive credit and whether it will be
considered for an award.
Thank you for your assistance in preparing this report.
Ima Student (your signature)\\

\newpage\thispagestyle{fancy}\sectionfont{\scshape}
\section*{Table of Contents}
\normalsize
\begin{enumerate}[label=\arabic*,leftmargin=*,labelsep=2ex,ref=\arabic*]
    \item Intorduction \dotfill 3
    \item Second chapter \dotfill 4
      \begin{enumerate}[label*=.\arabic*,leftmargin=*,labelsep=2ex]
        \item First section \dotfill 5
        \begin{enumerate}[label*=.\arabic*,leftmargin=*,labelsep=2ex]
        \item First sub section \dotfill 6
      \end{enumerate}
      \end{enumerate}
    \item References \dotfill 5
\end{enumerate}
\fancyfoot[C]{ii}

\newpage\thispagestyle{fancy}\sectionfont{\scshape}
\section*{List of Figures}
\normalsize\cfoot{3aa}
\begin{enumerate}[label=\arabic*,leftmargin=*,labelsep=2ex,ref=\arabic*]
    \item Awesome Graph \dotfill 3
\end{enumerate}

\fancyfoot[C]{iii}

\newpage\thispagestyle{fancy}\sectionfont{\scshape}
\section*{Executive Summary}
God this is bad.

\fancyfoot[C]{iv}

%-------------------------------------------------------------------------------
% BODY
%-------------------------------------------------------------------------------

\newpage\thispagestyle{fancy}\sectionfont{\scshape}

% Reset the page counter
\setcounter{page}{1}
\fancyfoot[C]{\thepage}

\section*{Introduction}
\addcontentsline{toc}{section}{Introduction}
\par\indent
This is the text in first paragraph. This is the text in first 
paragraph. This is the text in first paragraph.
\\ \par\noindent
This is the text in second paragraph. This is the text in second 
paragraph. This is the text in second paragraph.\\

\newpage\thispagestyle{fancy}\sectionfont{\scshape}
\section*{Analysis}
\addcontentsline{toc}{section}{Introduction}
\par\indent
This is the text in first paragraph. This is the text in first 
paragraph. This is the text in first paragraph.
\\ \par\noindent
This is the text in second paragraph. This is the text in second 
paragraph. This is the text in second paragraph.\\

\newpage\thispagestyle{fancy}\sectionfont{\scshape}
\section*{Conclusions}
\addcontentsline{toc}{section}{Introduction}
\par\indent
This is the text in first paragraph. This is the text in first 
paragraph. This is the text in first paragraph.
\\ \par\noindent
This is the text in second paragraph. This is the text in second 
paragraph. This is the text in second paragraph.\\

\newpage\thispagestyle{fancy}\sectionfont{\scshape}
\section*{Recommendations}
\addcontentsline{toc}{section}{Introduction}
\par\indent
This is the text in first paragraph. This is the text in first 
paragraph. This is the text in first paragraph.
\\ \par\noindent
This is the text in second paragraph. This is the text in second 
paragraph. This is the text in second paragraph.\\


%-------------------------------------------------------------------------------
% REFERENCES
%-------------------------------------------------------------------------------
\newpage
\section*{References}
\addcontentsline{toc}{section}{References}

Anand, U., 2010. The Elusive Free Radicals, \textit{The Clinical Chemist,} [e-journal] Available at:<\url{http://www.clinchem.org/content/56/10/1649.full.pdf}> [Accessed 2 November 2013]
\newline
\newline

Biology Forums, 2012. \textit{Normal glomerulus. Acute glomerulonephritis.} [online] Available at: <\url{http://biology-forums.com/index.php?action=gallery;sa=view;id=9284}> [Accessed 23 October 2013].
\newline
\newline

Budisavljevic, M., Hodge, L., Barber, K., Fulmer, J., Durazo-Arvizu, R., Self, S., Kuhlmann, M., Raymond, J. and Greene, E., 2003. Oxidative stress in the pathogenesis of experimental mesangial proliferative glomerulonephritis, \textit{American Journal of Physiology - Renal Physiology,} 285(6), pp. 1138-1148.
\newline
\newline


\end{document}

%-------------------------------------------------------------------------------
% SNIPPETS
%-------------------------------------------------------------------------------

%\begin{figure}[!ht]
%	\centering
%	\includegraphics[width=0.8\textwidth]{file_name}
%	\caption{}
%	\centering
%	\label{label:file_name}
%\end{figure}

%\begin{figure}[!ht]
%	\centering
%	\includegraphics[width=0.8\textwidth]{graph}
%	\caption{Blood pressure ranges and associated level of hypertension (American Heart Association, 2013).}
%	\centering
%	\label{label:graph}
%\end{figure}

%\begin{wrapfigure}{r}{0.30\textwidth}
%	\vspace{-40pt}
%	\begin{center}
%		\includegraphics[width=0.29\textwidth]{file_name}
%	\end{center}
%	\vspace{-20pt}
%	\caption{}
%	\label{label:file_name}
%\end{wrapfigure}

%\begin{wrapfigure}{r}{0.45\textwidth}
%	\begin{center}
%		\includegraphics[width=0.29\textwidth]{manometer}
%	\end{center}
%	\caption{Aneroid sphygmomanometer with stethoscope (Medicalexpo, 2012).}
%	\label{label:manometer}
%\end{wrapfigure}

%\begin{table}[!ht]\footnotesize
%	\centering
%	\begin{tabular}{cccccc}
%	\toprule
%	\multicolumn{2}{c} {Pearson's correlation test} & \multicolumn{4}{c} {Independent t-test} \\
%	\midrule	
%	\multicolumn{2}{c} {Gender} & \multicolumn{2}{c} {Activity level} & \multicolumn{2}{c} {Gender} \\
%	\midrule
%	Males & Females & 1st level & 6th level & Males & Females \\
%	\midrule
%	\multicolumn{2}{c} {BMI vs. SP} & \multicolumn{2}{c} {Systolic pressure} & \multicolumn{2}{c} {Systolic Pressure} \\
%	\multicolumn{2}{c} {BMI vs. DP} & \multicolumn{2}{c} {Diastolic pressure} & \multicolumn{2}{c} {Diastolic pressure} \\
%	\multicolumn{2}{c} {BMI vs. MAP} & \multicolumn{2}{c} {MAP} & \multicolumn{2}{c} {MAP} \\
%	\multicolumn{2}{c} {W:H ratio vs. SP} & \multicolumn{2}{c} {BMI} & \multicolumn{2}{c} {BMI} \\
%	\multicolumn{2}{c} {W:H ratio vs. DP} & \multicolumn{2}{c} {W:H ratio} & \multicolumn{2}{c} {W:H ratio} \\
%	\multicolumn{2}{c} {W:H ratio vs. MAP} & \multicolumn{2}{c} {\% Body fat} & \multicolumn{2}{c} {\% Body fat} \\
%	\multicolumn{2}{c} {} & \multicolumn{2}{c} {Height} & \multicolumn{2}{c} {Height} \\
%	\multicolumn{2}{c} {} & \multicolumn{2}{c} {Weight} & \multicolumn{2}{c} {Weight} \\
%	\multicolumn{2}{c} {} & \multicolumn{2}{c} {Heart rate} & \multicolumn{2}{c} {Heart rate} \\
%	\bottomrule
%	\end{tabular}
%	\caption{Parameters that were analysed and related statistical test performed for current study. BMI - body mass index; SP - systolic pressure; DP - diastolic pressure; MAP - mean arterial pressure; W:H ratio - waist to hip ratio.}
%	\label{label:tests}
%\end{table}
